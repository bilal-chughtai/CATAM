\documentclass[10pt,a4paper]{report}
\usepackage[utf8]{inputenc}
\usepackage{amsmath,mathtools}
\usepackage{amsfonts}
\usepackage{amssymb}
\usepackage{graphicx}
\usepackage{hyperref}
\usepackage{bm}
\usepackage{gensymb}
\usepackage{listings} 	
\usepackage[left=2cm,right=2cm,top=2cm,bottom=2cm]{geometry}
\setlength\parindent{0pt}
\graphicspath{{./images/}}
\newcommand{\legendre}[2]{(\frac{#1}{#2})}


\usepackage[english]{babel}
 
\usepackage{amsthm}
 
\newtheorem*{lemma}{Lemma}
\newtheorem*{theorem}{Theorem}
\begin{document}


\textbf{CATAM Part II - 17.3 - Hamiltonian Cycles}
\thispagestyle{empty}

\newpage

\subsection*{Introduction}

Our first task is to create a sensible data structure to store our graph in, and also write a function that generates a random graph from $\mathcal{G}(n,p)$. Since I'm coding in python, I've implemented a Graph class, which stores all our graph's information, and will contain various methods pertaining to our graph. It allows for us to generate a random graph or input a vertex and edge set.

\subsection*{Question 1}	

The simplest, and least efficient way to check if a graph has a Hamiltonian cycle is to check each of the $n!$ individual permutations of the vertex set. We can reduce checking slightly since cyclic permutations are equivalent, sow we only have $(n-1)!$. The hard part here is iterating over all permutations, since theres eventually too many to store in memory. We can overcome this by, say, Heap's Algorithm\footnote{\url{https://en.wikipedia.org/wiki/Heap\%27s_algorithm}}, but our implementation uses the $permutations$ iterator from the standard python library $itertools$.  It can be found under $naive_hamiltonian()$ as a method of the Graph class. The following data is gathered by $q1()$. TODO maybe implement own iterator

\begin{lstlisting}[breaklines]
* Order=3, Size=3, {1: {2, 3}, 2: {1, 3}, 3: {1, 2}} * has a hamiltonian cycle: [1,2,3] 
* Order=3, Size=2, {1: {2}, 2: {1, 3}, 3: {2}} * has no hamiltonian cycle
* Order=4, Size=3, {1: {2, 3}, 2: {1, 3}, 3: {1, 2}} * has no hamiltonian cycle
\end{lstlisting}


\begin{table}[h]
\centering
\begin{tabular}{|l|l|l|l|l|l|l|l|l|l|l|}
\hline
\textbf{p/n}         &  \textbf{4} & \textbf{6} & \textbf{8} & \textbf{10} & \textbf{12} & \textbf{14} & \textbf{16} & \textbf{18} & \textbf{20} \\ \hline
\textbf{0.1}     & 985        & 1075       & 978        & 1009        & 938         & 1059        & 1033        & 975         & 1020        \\ \hline
\textbf{0.3}       & 3036       & 3019       & 2966       & 3048        & 3029        & 2954        & 2968        & 2968        & 3075        \\ \hline
\textbf{0.5}        & 4911       & 5022       & 4933       & 5077        & 4997        & 5026        & 4894        & 4953        & 5026        \\ \hline
\textbf{0.7}        & 6916       & 6978       & 7083       & 6996        & 7009        & 6994        & 7009        & 6983        & 7088        \\ \hline
\textbf{0.9}        & 9025       & 8990       & 9000       & 8954        & 8982        & 9021        & 8959        & 9040        & 9025        \\ \hline
\end{tabular}
\caption{Number of graphs containing Hamiltonian cycles from a selection of 10000 taken from $\mathcal{G}(n,p)$}
\end{table}

\begin{table}[h]
\centering
\begin{tabular}{|l|l|l|l|l|l|l|l|l|l|l|}
\hline
\textbf{a/n}   & \textbf{4} & \textbf{6} & \textbf{8} & \textbf{10} & \textbf{12} & \textbf{14} & \textbf{16} & \textbf{18} & \textbf{20} \\ \hline
\textbf{0.1}         & 332        & 293        & 262        & 246         & 216         & 200         & 168         & 164         & 150         \\ \hline
\textbf{0.55}       & 1838       & 1673       & 1494       & 1274        & 1158        & 1087        & 910         & 887         & 828         \\ \hline
\textbf{1}           & 3484       & 2998       & 2509       & 2216        & 2149        & 1912        & 1715        & 1625        & 1544        \\ \hline
\textbf{1.45}        & 4925       & 4330       & 3734       & 3305        & 3148        & 2730        & 2512        & 2366        & 2195        \\ \hline
\textbf{1.9}         & 6583       & 5685       & 4924       & 4406        & 4042        & 3646        & 3385        & 3067        & 2871        \\ \hline
\end{tabular}
\caption{Number of graphs containing Hamiltonian cycles from a selection of 10000 taken from $\mathcal{G}(n,a\log{n}/n)$}
\label{tab:my-table}
\end{table}

\subsection*{Question 2}
Suppose we have n vertices. The worst case is we have no Hamiltonian cycle, so each of the $(n-1)!$ permutations is checked. We assume each simple operation, ie iterating and looking up a value in a list take time $c$. For each. At worst, we have the first $n-1$ entries of our permutation form a Hamiltonian path \footnote{This can't happen for each but its an ok approximation}, so we do $2n$ lookups and $n$ iterations, giving us a time of $3nc$ per permutation. Supposing there is a time $d$ taken to iterate our permutation, we end up with a total running time of $(n-1)!(3nc+d) = O(n!)$.\\

Now for an average case, supposing an average graph can be taken from $\mathcal{G}(n,1/2)$. Supposing such an average graph has $k$ Hamiltonian cycles, from Table 1 it seems reasonable to assume $k=1/2$. So in the case we have a Hamiltonian cycle, we expect to find it in $(n-1)!/2$ permutations, with each prior permutation failing in $n/2$ steps. This gives a running time of $(n-1)!(3nc+d)/4$. In the other case where we don't have a Hamiltonian cycle, we similarly get $(n-1)!(3nc+d)/2$, so expect a running time of $3(n-1)!(3nc+d)/4 = O(n!)$. \\

So the algorithm cannot handle too large an n.

\subsection*{Question 3}
Modifying the code slightly, to print the number of graphs G with $\delta(G) < 2$, we find most, at least for $a<1.45$ for a defined in Table 2, graphs fail to be Hamiltonian because  $\delta(G) < 2$. The second range may well have been chosen because of the following theorem, proved in IID Graph Theory.

\begin{theorem}
Let $\omega(n) \rightarrow \infty$.  If $p =\frac{\log(n)-\omega(n)}{n}$ then G has isolated vertices almost surely.  If $p =\frac{\log(n)+\omega(n)}{n}$ then G has no isolated vertices almost surely.
\end{theorem}

\begin{proof}
Note if $X=\Sigma_A I_A$ is a sum of indicator functions then $\text{Var}(X)=\Sigma_{A,B}\mathbb{P}(A)[ \mathbb{P}(B \mid A) -\mathbb{P}(B)]$ simply by expanding.\\

Now let X be the number of isolated vertices $X=\Sigma_v I_v$ where $I_v$ indicated v being isolated. So 

\begin{align*}
\text{Var}(X)&=\Sigma_{u,v} \mathbb{P}(u \text{ isolated})[ \mathbb{P}(v u \text{ isolated} \mid u u \text{ isolated}) -\mathbb{P}(v \text{ isolated})]\\
&=(1-p)^{n-1}[1-(1-p)^{n-1}] + n(n-1)(1-p)^{n-1}[(1-p)^{n-2}-(1-p)^{n-1}]\\
&\leq \mathbb{E}(X)+n^2(1-p)^{n-1}(1-p)^{n-2}\\
&= \mathbb{E}(X) + \frac{p}{1-p}(\mathbb{E}(X))^2
\end{align*}

where the first term comes from u and v being the same, and the second term otherwise. Now if $p =\frac{\log(n)+\omega(n)}{n}$ 

\begin{align*}
\mathbb{E}(X) = \frac{1}{1-p}n(1-p)^n \leq \frac{1}{1-p}ne^{-pn} \rightarrow 0
\end{align*}

So $X=0$ a.s. by Markov's Inequality. If $p =\frac{\log(n)-\omega(n)}{n}$ 

\begin{align*}
\mathbb{E}(X) \approx \frac{1}{1-p}ne^{-pn} \rightarrow \infty
\end{align*}

So 

\begin{align*}
\frac{\text{Var}(X)}{(\mathbb{E}(X))^2} \leq \frac{1}{\mathbb{E}(X)} + \frac{p}{1-p} \rightarrow 0
\end{align*}

So $X\neq0$ a.s. by Chebyshev's Inequality.

\end{proof}

Here we're considering a range of values $p\in[\frac{\log(n)-0.9\log(n)}{n},\frac{\log(n)+0.9\log(n)}{n} ]$, so as n grows, the probability of having isolated vertices for small a grows tends to 1, which agrees with our results.


\end{document}